 %!TeX root = Template.tex
\documentclass{article}
\usepackage[dvipsnames, svgnames, x11names]{xcolor}
\usepackage{tikz}
\usepackage{pgfplots}
\usepackage{setspace}
\usepackage{units}
\usepackage{graphicx}
\usepackage{amsopn}
\usepackage{bbding}
\usepackage{amsmath}
\usepackage{hyperref}
\usepackage{cancel}
\usepackage{gensymb}
\usepackage[margin = 1.35in]{geometry}
\AtBeginEnvironment{document}{\everymath{\displaystyle}}
\title{MATH 1C Notecard}
\date{10/3/2024}
\author{Tejas Patel}
\begin{document}
\maketitle
\section*{15.1 Double Integrals over Rectangles}
The single variable definite integral: $\int_{a}^{b} f(x) dx = \lim_{n\to\infty} \sum_{i=1}^{n} f(x^*_i) \Delta x$
\\ Volume as a Double Riemann Sum: $V \approx \sum_{i=1}^{m}\sum_{j=1}^{n} f(x^*_{ij}, y^*_{ij}) \Delta A$
\\ Volume as a Double Integral: $\int\int_{R}f(x,y) dA = \sum_{i=1}^{m}\sum_{j=1}^{n} f(x^*_{ij}, y^*_{ij}) \Delta A$
\subsection*{Fubini's Theorem}
If $f$ is continuous on the rectangle $R=\{(x,y) \; | \; a \leq x \leq b,c\leq y \leq d\}$
\\[0.1in]then it is known $\iint\limits_{R} f(x,y)dA = \int_{a}^{b}\int_{c}^{d}f(x,y)\; dy \; dx = \int_{c}^{d}\int_{a}^{b} f(x,y) \; dx \; dy$
\section*{15.2 Double Integrals over General Regions}
\section*{15.3 Double Integrals in Polar Coordinates}
\section*{15.4 Applications of Double Integrals}
\section*{15.5 Surface Area}
\section*{15.6 Triple Integrals}
\section*{15.7 Triple Integrals in Cylndrical Coordinates}
\section*{15.8 Triple Integrals in Spherical Coordinates}
\section*{15.9 Change of Variables in Multiple Integrals}


\end{document}