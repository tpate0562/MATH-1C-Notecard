 %!TeX root = Template.tex
\documentclass{article}
\usepackage[dvipsnames, svgnames, x11names]{xcolor}
\usepackage{tikz}
\usepackage{pgfplots}
\usepackage{setspace}
\usepackage{units}
\usepackage{graphicx}
\usepackage{amsopn}
\usepackage{bbding}
\usepackage{amsmath}
\usepackage{hyperref}
\usepackage{cancel}
\usepackage{gensymb}
\usepackage[margin = 1.35in]{geometry}
\AtBeginEnvironment{document}{\everymath{\displaystyle}}
\title{MATH 1C Test 2 Notecard}
\date{10/3/2024}
\author{Tejas Patel}
\begin{document}
\maketitle
\section*{15.1 Double Integrals over Rectangles}
The single variable definite integral: $\int_{a}^{b} f(x) dx = \lim_{n\to\infty} \sum_{i=1}^{n} f(x^*_i) \Delta x$
\\ Volume as a Double Riemann Sum: $V \approx \sum_{i=1}^{m}\sum_{j=1}^{n} f(x^*_{ij}, y^*_{ij}) \Delta A$
\\ Volume as a Double Integral: $\int\int_{R}f(x,y) dA = \sum_{i=1}^{m}\sum_{j=1}^{n} f(x^*_{ij}, y^*_{ij}) \Delta A$
\subsection*{Fubini's Theorem}
If $f$ is continuous on the rectangle $R=\{(x,y) \; | \; a \leq x \leq b,c\leq y \leq d\}$
\\[0.1in]then it is known $\iint\limits_{R} f(x,y)dA = \int_{a}^{b}\int_{c}^{d}f(x,y)\; dy \; dx = \int_{c}^{d}\int_{a}^{b} f(x,y) \; dx \; dy$
\subsection*{Special Case}
In the special case that $f(x,y)$ can be factored as a product of a function of $x$ only and a function of $y$ only then it is known the following is true:
\\[0.1in] $\iint\limits_{R} g(x)\,h(y)\,dA =\int_{a}^{b}g(x)\; dx\, \int_{c}^{d} h(y)\; dy$
\subsection*{Average Value}
The average value of a function $f$ of two variables defined on a rectangle R is:
\\$$ f_{avg}=\frac{1}{A(R)} \iint\limits_{R} f(x,y)dA$$ where A(R) is the area of R.
\pagebreak
\section*{15.2 Double Integrals over General Regions}
$$
F(x,y)=\begin{cases}
			f(x,y) & \text{if $(x,y)$ s in }D\\
            0 & \text{if $(x,y)$ is in $R$ but not in }D
		 \end{cases}
$$
$$\iint\limits_{D} f(x,y)\; dA =\iint\limits_{D} F(x,y)\; dA $$
If $f$ is continuous on a Type I region $D$ such that the fixed bounds are vertical, \\$D=\{(x,y)\; | \; a\leq x \leq b,\; g_1(x)\leq y \leq g_2(x)\}$, then 
$$\iint\limits_{D} f(x,y) \; dA = \int_{a}^{b} \int_{g_1(x)}^{g_2(x)} f(x,y) \; dy \; dx$$
\\If $f$ is continuous on a Type II region $D$ such that the fixed bounds are horizontal, \\$D=\{(x,y)\; | \; c\leq x \leq d,\; h_1(x)\leq y \leq h_2(x)\}$, then 
$$\iint\limits_{D} f(x,y) \; dA = \int_{a}^{b} \int_{h_1(x)}^{h_2(x)} f(x,y) \; dx \; dy$$
\section*{15.3 Double Integrals in Polar Coordinates}
$$\iint\limits_{R} f(x,y) \; dA = \int_{\alpha}^{\beta} \int_{a}^{b} f(r\cos \theta,  r\sin \theta)\, r \; dr \; d\theta$$
\textbf{Put all half angle and double angle identities on notecard}
If $f$ is continuous on a polar region of the form $$D\; = \; \{ (r,\theta )|\alpha \leq \theta \leq \beta , \, h_1(\theta) \leq r \leq h_2(\theta)\}$$

\section*{15.4 Applications of Double Integrals}
\begin{center}
	Electric Charge
\end{center}
$$Q= \iint\limits_{D} \sigma(x,y) \; dA$$
\begin{center} {Center of Mass of a Lamina}\end{center}[0.1in]
$\bar{x} = \frac{M_y}{m} = \frac{1}{m} \iint\limits_D x\rho (x,y)\; dA\;\;\;$ and $\;\;\; \bar{y} = \frac{M_x}{m} = \frac{1}{m} \iint\limits_D y\rho (x,y)\; dA$
where $$m = \iint\limits_D \rho (x,y)\; dA$$
\pagebreak
\section*{15.5 Surface Area}
The surface area with equation $z=f(x,y),(x,y)\epsilon D$ where $f_x$ and $f_y$ are continuous is
$$A(S) = \iint\limits_D \sqrt{[f_x(x,y)]^2+[f_y(x,y)]^2 + 1}\; dA$$
$$A(S) = \iint\limits_D \sqrt{\left[\frac{\partial z}{\partial x}\right]^2+\left[\frac{\partial y}{\partial z}\right]^2 + 1}\; dA$$
For the noteacrd:
$$A(S) = \iint\limits_D \sqrt{\left[f_x\right]^2+\left[f_y\right]^2 + 1}\; dA$$
\section*{15.6 Triple Integrals}
\section*{15.7 Triple Integrals in Cylndrical Coordinates}
\section*{15.8 Triple Integrals in Spherical Coordinates}
\section*{15.9 Change of Variables in Multiple Integrals}


\end{document}